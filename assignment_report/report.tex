%% This is file `elsarticle-template-1-num.tex',
%%
%% Copyright 2009 Elsevier Ltd
%%
%% This file is part of the 'Elsarticle Bundle'.
%% ---------------------------------------------
%%
%% It may be distributed under the conditions of the LaTeX Project Public
%% License, either version 1.2 of this license or (at your option) any
%% later version.  The latest version of this license is in
%%    http://www.latex-project.org/lppl.txt
%% and version 1.2 or later is part of all distributions of LaTeX
%% version 1999/12/01 or later.
%%
%% Template article for Elsevier's document class `elsarticle'
%% with numbered style bibliographic references
%%
%% $Id: elsarticle-template-1-num.tex 149 2009-10-08 05:01:15Z rishi $
%% $URL: http://lenova.river-valley.com/svn/elsbst/trunk/elsarticle-template-1-num.tex $
%%
\documentclass[final,12pt,a4paper]{elsarticle}

%% Use the option review to obtain double line spacing
%% \documentclass[preprint,review,12pt]{elsarticle}

%% Use the options 1p,twocolumn; 3p; 3p,twocolumn; 5p; or 5p,twocolumn
%% for a journal layout:
%% \documentclass[final,1p,times]{elsarticle}
%% \documentclass[final,1p,times,twocolumn]{elsarticle}
%% \documentclass[final,3p,times]{elsarticle}
%% \documentclass[final,3p,times,twocolumn]{elsarticle}
%% \documentclass[final,5p,times]{elsarticle}
%% \documentclass[final,5p,times,twocolumn]{elsarticle}

%% The graphicx package provides the includegraphics command.
\usepackage{graphicx}
%% The amssymb package provides various useful mathematical symbols
\usepackage{amssymb}
%% The amsthm package provides extended theorem environments
%% \usepackage{amsthm}

\usepackage{booktabs}
\usepackage{xcolor}
\usepackage{sourcecodepro}
\usepackage{url}
\usepackage{listings}
\usepackage[utf8]{inputenc}
\usepackage[english]{babel}
\usepackage{multirow}
\usepackage{textcomp}
\usepackage{caption}

\definecolor{Accent}{HTML}{157FFF}

\lstdefinestyle{customMtheme}{%
  backgroundcolor={},
  basicstyle=\ttfamily\scriptsize,
  breakatwhitespace=true,
  breaklines=true,
  captionpos=n,
  commentstyle=\color{orange},
  escapeinside={\%*}{*)},
  extendedchars=true,
  frame=n,
  keywordstyle=\color{Accent},
  language=C++,
  rulecolor=\color{black},
  showspaces=false,
  showstringspaces=false,
  xleftmargin=.5cm,
  xrightmargin=.5cm,
  showtabs=false,
  stepnumber=2,
  stringstyle=\color{gray},
  tabsize=4,
  keywords={void, int, float, main,
  if, else, malloc, NULL,
  fprintf, stderr, for, make, gcc, o, Enter, Ctrl},
  otherkeywords={\#pragma, \#include, \&, \*, +, -, /, [, ], >, <, \$, \., std\=c11}
}
\lstset{basicstyle=\ttfamily\scriptsize,style=customMtheme}

\renewcommand*{\UrlFont}{\ttfamily\scriptsize\relax}

\graphicspath{{./img/}}

%% The lineno packages adds line numbers. Start line numbering with
%% \begin{linenumbers}, end it with \end{linenumbers}. Or switch it on
%% for the whole article with \linenumbers after \end{frontmatter}.
%% \usepackage{lineno}

%% natbib.sty is loaded by default. However, natbib options can be
%% provided with \biboptions{...} command. Following options are
%% valid:

%%   round  -  round parentheses are used (default)
%%   square -  square brackets are used   [option]
%%   curly  -  curly braces are used      {option}
%%   angle  -  angle brackets are used    <option>
%%   semicolon  -  multiple citations separated by semi-colon
%%   colon  - same as semicolon, an earlier confusion
%%   comma  -  separated by comma
%%   numbers-  selects numerical citations
%%   super  -  numerical citations as superscripts
%%   sort   -  sorts multiple citations according to order in ref. list
%%   sort&compress   -  like sort, but also compresses numerical citations
%%   compress - compresses without sorting
%%
%% \biboptions{comma,round}

% \biboptions{}

%% Removing lines when no abstract is given
\makeatletter
\renewcommand{\MaketitleBox}{%
    \resetTitleCounters
        \def\baselinestretch{1}%
        \begin{center}
    \def\baselinestretch{1}%
        \Large \@title \par
        \vskip 18pt
        \normalsize\elsauthors \par
        \vskip 10pt
        \footnotesize \itshape \elsaddress \par
        \end{center}
    \vskip 12pt
}
\makeatother

%% Removing custom footer on fist page
\makeatletter
\def\ps@pprintTitle{%
    \let\@oddhead\@empty
        \let\@evenhead\@empty
        \def\@oddfoot{\centerline{\thepage}%
        }%
    \let\@evenfoot\@oddfoot
}%
\makeatother

\journal{Computação Móvel}

\begin{document}

\begin{frontmatter}

%% Title, authors and addresses

\title{EP3 MAC5743-MAC0463 \\ Evaluating Mobility, Communication \\ and Routing with The ONE Framework}

%% use the tnoteref command within \title for footnotes;
%% use the tnotetext command for the associated footnote;
%% use the fnref command within \author or \address for footnotes;
%% use the fntext command for the associated footnote;
%% use the corref command within \author for corresponding author footnotes;
%% use the cortext command for the associated footnote;
%% use the ead command for the email address,
%% and the form \ead[url] for the home page:
%%
%% \title{Title\tnoteref{label1}}
%% \tnotetext[label1]{}
%% \author{Name\corref{cor1}\fnref{label2}}
%% \ead{email address}
%% \ead[url]{home page}
%% \fntext[label2]{}
%% \cortext[cor1]{}
%% \address{Address\fnref{label3}}
%% \fntext[label3]{}


%% use optional labels to link authors explicitly to addresses:
%% \author[label1,label2]{<author name>}
%% \address[label1]{<address>}
%% \address[label2]{<address>}

\author{Lucas Morais, Lucas Kanashiro and Pedro Bruel}

\address{Instituto de Matemática e Estatística - Universidade de São Paulo
(USP) \\ Rua do Matão, 1010 - CEP 05508-090 - São Paulo - SP
}

%%\begin{abstract}
%% Text of abstract
%% Suspendisse potenti. Suspendisse quis sem elit, et mattis nisl. Phasellus
%% consequat erat eu velit rhoncus non pharetra neque auctor. Phasellus eu lacus
%% quam. Ut ipsum dolor, euismod aliquam congue sed, lobortis et orci. Mauris eget
%% velit id arcu ultricies auctor in eget dolor. Pellentesque suscipit adipiscing
%% sem, imperdiet laoreet dolor elementum ut. Mauris condimentum est sed velit
%% lacinia placerat. Vestibulum ante ipsum primis in faucibus orci luctus et
%% ultrices posuere cubilia Curae; Nullam diam metus, pharetra vitae euismod sed,
%% placerat ultrices eros. Aliquam tincidunt dapibus venenatis. In interdum tellus
%% nec justo accumsan aliquam. Nulla sit amet massa augue.
%% \end{abstract}
%%
%% \begin{keyword}
%% Science \sep Publication \sep Complicated
%% keywords here, in the form: keyword \sep keyword

%% MSC codes here, in the form: \MSC code \sep code
%% or \MSC[2008] code \sep code (2000 is the default)

%% \end{keyword}

\end{frontmatter}

%%
%% Start line numbering here if you want
%%
%% \linenumbers

%% main text
\section{Introduction}

\section{VANET-like Scenario}

While designing the following experiments, we aimed to simulate how DTNs based
on several different routing protocols would behave while serving hosts that
moved like cars. In such scenario, hosts travel along the roads of an accurate
model of Finland's capital, Helsinki, with transmission range set to 50m and
speed varying from 30 to 60 kilometers per hour. Such values were chosen so as
to be representative of the behavior of real cars in a city environment and of
the features that could expect from wireless transmission protocols to be
rolled out in the following years.

\subsection{Experimental Settings}

Tests were described by 3 configuration files generating 30 different parameter
combinations each.  All tests were run once with the same random seed.  The
following tables describe the relevant parameters explored in this scenario.
The DTN performance for each particular configuration was measured according to
a single performance scale that we discuss next.
\paragraph{Performance Scale}

The experiments that we described should be useful for characterizing the
situations in which DTNs could be successfully used as VANETs. That being the
case, we found the need for developing a single performance for uniformly
judging the gathered results, which can be defined as follows. For every
transmission scenario $T_i$ there are measurements for Message Delivery
Probability ($P_i$), Average Latency ($L_i$) and Overhead Ratio ($O_i$):

\begin{itemize}
    \item The Message Delivery Probability is the chance that a message
    created withing simulation time will be delivered to the target host
    before the simulation times out

    \item The Average Latency is the average amount of time, in seconds,
    that it took for the successfully delivered messages to be delivered

    \item The Overhead Ratio is the ratio between the number of packet
    transmissions and the number of messages created by the whole system.
    This value will always be in $[1, \infty[$, where lower figures correspond
    to more efficient use of the transmission channels
\end{itemize}

Based on that, every scenario $T_i$ is assigned a quality grade $Q_i = (p_i,
l_i, o_i) = (\left \lfloor{P_i * 20}\right \rfloor, - log_{2}(L_i), - \left \lfloor{O_i}\right \rfloor)$.
We say that $T_i$ is better than $T_j$ if:

\begin{enumerate}
    \item $p_i > p_j$ or
    \item $p_i = p_j$ and $l_i > l_j$ or
    \item $p_i = p_j$ and $l_i = l_j$ and $o_i > o_j$
\end{enumerate}

We believe this is a meaningful performance scale because Delivery
Probability is the most important performance metric we have under analysis,
followed by Average Latency and Overhead Ratio.

\paragraph{Parameters}

Table \ref{tab:params} lists the parameters used for each configuration
in the experiments.

\begin{table}[htpb]
\centering
    \begin{tabular}{@{}p{0.46\textwidth}p{0.46\textwidth}@{}}
        \toprule
        \multicolumn{2}{c}{\textit{Experimental Configurations}} \\ \midrule
        \textit{Transmission Range} &  50 m \\
        \addlinespace
        \textit{Avg. Packet Creation Time} & 30 s  \\
        \addlinespace
        \textit{Transmission Speed} & 128 KB/s, 1 MB/s \\
        \addlinespace
        \textit{Protocols} & SprayAndWaitRouter, DirectDeliveryRouter, EpidemicRouter, EpidemicOracleRouter \\
        \addlinespace
        \textit{Hosts} &  32, 256, 2048 \\
        \addlinespace
        \textit{Buffer Size} & 1 M, 64 M, 2048 M \\
        \addlinespace
        \textit{Packet TTL} &  5 h \\
        \addlinespace
        \textit{Simulation Time} &  50 min\\ \bottomrule
    \end{tabular}
    \caption{Parameters in tested configurations}
    \label{tab:params}
\end{table}

\subsection{Results}

Tables \ref{tab:res1} and \ref{tab:res2} show the results of all the 72 tests that were conducted
for the scenario above, ordered according to the performance scale described
from best to worst.

\begin{table}[htpb]
\centering
\begin{tabular}{@{}p{0.02\textwidth}p{0.3\textwidth}p{0.06\textwidth}p{0.08\textwidth}p{0.13\textwidth}p{0.04\textwidth}p{0.06\textwidth}p{0.1\textwidth}@{}}
\toprule
\textnumero & Routing Protocol & Hosts & Buffer & T. Speed & P & L & O \\ \midrule
0 & EpidemicOracleRouter & 2048 & 2048M & 128k & 20 & -5 & -3005 \\
1 & EpidemicOracleRouter & 2048 & 2048M & 1M & 20 & -5 & -3005 \\
2 & EpidemicOracleRouter & 2048 & 64M & 128k & 20 & -5 & -37272 \\
3 & EpidemicOracleRouter & 2048 & 64M & 1M & 20 & -5 & -37272 \\
4 & EpidemicRouter & 2048 & 2048M & 1M & 18 & -7 & -2161 \\
5 & EpidemicRouter & 2048 & 64M & 1M & 18 & -7 & -2843 \\
6 & EpidemicRouter & 256 & 2048M & 1M & 16 & -8 & -252 \\
7 & EpidemicRouter & 256 & 64M & 1M & 16 & -8 & -253 \\
8 & EpidemicOracleRouter & 256 & 2048M & 128k & 16 & -8 & -353 \\
9 & EpidemicOracleRouter & 256 & 2048M & 1M & 16 & -8 & -353 \\
10 & EpidemicOracleRouter & 256 & 64M & 128k & 16 & -8 & -363 \\
11 & EpidemicOracleRouter & 256 & 64M & 1M & 16 & -8 & -363 \\
12 & EpidemicRouter & 2048 & 2048M & 128k & 14 & -9 & -2042 \\
13 & EpidemicRouter & 2048 & 64M & 128k & 14 & -9 & -2042 \\
14 & SprayAndWaitRouter & 256 & 64M & 1M & 10 & -9 & -9 \\
15 & SprayAndWaitRouter & 256 & 2048M & 1M & 10 & -9 & -9 \\
16 & EpidemicRouter & 256 & 64M & 128k & 9 & -9 & -174 \\
17 & EpidemicRouter & 256 & 2048M & 128k & 9 & -9 & -174 \\
18 & SprayAndWaitRouter & 256 & 1M & 1M & 7 & -9 & -13 \\
19 & EpidemicOracleRouter & 32 & 64M & 1M & 7 & -10 & -29 \\
20 & EpidemicOracleRouter & 32 & 2048M & 1M & 7 & -10 & -29 \\
21 & EpidemicOracleRouter & 32 & 64M & 128k & 7 & -10 & -29 \\
22 & EpidemicOracleRouter & 32 & 2048M & 128k & 7 & -10 & -29 \\
23 & SprayAndWaitRouter & 2048 & 2048M & 1M & 6 & -9 & -14 \\
24 & SprayAndWaitRouter & 256 & 64M & 128k & 6 & -9 & -14 \\
25 & SprayAndWaitRouter & 256 & 2048M & 128k & 6 & -9 & -14 \\
26 & SprayAndWaitRouter & 2048 & 64M & 1M & 6 & -9 & -14 \\
27 & SprayAndWaitRouter & 2048 & 1M & 1M & 6 & -9 & -15 \\
28 & SprayAndWaitRouter & 32 & 64M & 1M & 6 & -10 & -9 \\
29 & SprayAndWaitRouter & 32 & 2048M & 1M & 6 & -10 & -9 \\
30 & EpidemicRouter & 32 & 64M & 1M & 6 & -10 & -12 \\
31 & EpidemicRouter & 32 & 2048M & 1M & 6 & -10 & -12 \\
32 & SprayAndWaitRouter & 2048 & 64M & 128k & 5 & -10 & -19 \\
33 & SprayAndWaitRouter & 2048 & 2048M & 128k & 5 & -10 & -19 \\
34 & SprayAndWaitRouter & 2048 & 1M & 128k & 4 & -10 & -23 \\
35 & DirectDeliveryRouter & 32 & 2048M & 1M & 3 & -9 & 0 \\ \bottomrule
\end{tabular}
\caption{First part of ordered results}
\label{tab:res1}
\end{table}

\begin{table}[htpb]
\centering
\begin{tabular}{@{}p{0.02\textwidth}p{0.3\textwidth}p{0.06\textwidth}p{0.08\textwidth}p{0.13\textwidth}p{0.04\textwidth}p{0.06\textwidth}p{0.1\textwidth}@{}}
\toprule
\textnumero & Routing Protocol & Hosts & Buffer & T. Speed & P & L & O \\ \midrule
36 & DirectDeliveryRouter & 2048 & 2048M & 1M & 3 & -9 & 0 \\
37 & DirectDeliveryRouter & 32 & 64M & 1M & 3 & -9 & 0 \\
38 & DirectDeliveryRouter & 2048 & 1M & 1M & 3 & -9 & 0 \\
39 & DirectDeliveryRouter & 2048 & 64M & 1M & 3 & -9 & 0 \\
40 & EpidemicRouter & 32 & 64M & 128k & 3 & -10 & -7 \\
41 & SprayAndWaitRouter & 32 & 2048M & 128k & 3 & -10 & -7 \\
42 & SprayAndWaitRouter & 32 & 64M & 128k & 3 & -10 & -7 \\
43 & EpidemicRouter & 32 & 2048M & 128k & 3 & -10 & -7 \\
44 & DirectDeliveryRouter & 2048 & 64M & 128k & 2 & -9 & 0 \\
45 & DirectDeliveryRouter & 256 & 1M & 1M & 2 & -9 & 0 \\
46 & DirectDeliveryRouter & 256 & 2048M & 1M & 2 & -9 & 0 \\
47 & DirectDeliveryRouter & 2048 & 1M & 128k & 2 & -9 & 0 \\
48 & DirectDeliveryRouter & 256 & 64M & 1M & 2 & -9 & 0 \\
49 & DirectDeliveryRouter & 2048 & 2048M & 128k & 2 & -9 & 0 \\
50 & SprayAndWaitRouter & 256 & 1M & 128k & 2 & -9 & -33 \\
51 & EpidemicOracleRouter & 256 & 1M & 128k & 2 & -9 & -376 \\
52 & EpidemicOracleRouter & 256 & 1M & 1M & 2 & -9 & -376 \\
53 & EpidemicRouter & 256 & 1M & 1M & 2 & -9 & -532 \\
54 & DirectDeliveryRouter & 32 & 1M & 1M & 1 & -7 & 0 \\
55 & SprayAndWaitRouter & 32 & 1M & 1M & 1 & -8 & -14 \\
56 & DirectDeliveryRouter & 256 & 2048M & 128k & 1 & -9 & 0 \\
57 & DirectDeliveryRouter & 256 & 1M & 128k & 1 & -9 & 0 \\
58 & DirectDeliveryRouter & 256 & 64M & 128k & 1 & -9 & 0 \\
59 & EpidemicRouter & 2048 & 1M & 1M & 1 & -9 & -60068 \\
60 & DirectDeliveryRouter & 32 & 2048M & 128k & 1 & -10 & 0 \\
61 & DirectDeliveryRouter & 32 & 64M & 128k & 1 & -10 & 0 \\
62 & EpidemicOracleRouter & 32 & 1M & 128k & 0 & -7 & -39 \\
63 & EpidemicOracleRouter & 32 & 1M & 1M & 0 & -7 & -39 \\
64 & EpidemicOracleRouter & 2048 & 1M & 1M & 0 & -7 & -52720 \\
65 & EpidemicOracleRouter & 2048 & 1M & 128k & 0 & -7 & -52720 \\
66 & DirectDeliveryRouter & 32 & 1M & 128k & 0 & -8 & 0 \\
67 & SprayAndWaitRouter & 32 & 1M & 128k & 0 & -8 & -33 \\
68 & EpidemicRouter & 32 & 1M & 1M & 0 & -8 & -36 \\
69 & EpidemicRouter & 32 & 1M & 128k & 0 & -8 & -81 \\
70 & EpidemicRouter & 256 & 1M & 128k & 0 & -9 & -952 \\
71 & EpidemicRouter & 2048 & 1M & 128k & 0 & -9 & -40119 \\ \bottomrule
\end{tabular}
\caption{Second part of ordered results}
\label{tab:res2}
\end{table}

\subsection{Discussion}

Only the EpidemicOracleRouter and the EpidemicRouter protocols are able to
display Delivery Probabilities higher than 70\% for any simulation scenario.
EpidemicOracleRouter and EpidemicRouter are usually much more taxing on the
transmission network than all other routing protocols: if scenarios based on
SprayAndWaitRouter and SprayAndWaitRouter never display overhead ratios above
33 (which occurs for experiment \#50), the same metric is as high as 3005 for the
best performing EpidemicOracleRouter scenario or as high as 37272 for other
top five scenarios using the same protocol, with EpidemicRouter scenarios
showing similar figures.

When all other parameters are fixed, increasing buffer size will usually improve delivery probability.
For the smallest buffer size, EpidemicRouter and EpidemicOracleRouter protocols
perform the worst, while SprayAndWaitRouter gives the best results.
For the EpidemicRouter protocol, increasing the transmission speed from 128K to
1M while maintaining all other parameters always improved delivery probability
(see, for example, \#12 vs \#4 and \#16 vs \#7).
For the EpidemicOracleRouter protocol, increasing the transmission speed from
128K to 1M while maintaining all other parameters never improved delivery
probability (see, for example, \#0 vs \#1 and \#10 vs \#11).
With all other parameters kept stable, changing the routing protocol from
EpidemicRouter to EpidemicOracleRouter would usually improve delivery
probability (see, for example, \#30 vs \#19 and \#4 vs \#1)
With transmission speed set to 1 Mbps and buffer size set to 1MB, changing the
routing protocol from either EpidemicRouter or EpidemicOracleRouter to
DirectDeliveryRouter always lead to higher delivery probability or much lower
overhead ratio with similar delivery probability (see, for example, \#63 vs
\#54 and \#59 vs \#38)
With buffer size set to 1MB, changing the protocol from either
EpidemicRouter, EpidemicOracleRouter or DirectDeliveryRouter to
SprayAndWaitRouter would usually lead to much higher delivery probability (see,
for example, \#52 vs \#18 or \#47 vs \#34)

Since EpidemicRouter and EpidemicOracleRouter are protocols that rely heavily
on flooding - not bothering too much with keeping resource utilization down -
these protocols should usually lead to the highest delivery probabilities and
lowest latencies, as facts (1) and (2) show. This also explains why lower
buffer sizes tend to disfavor these protocols, as facts (3), (4), (8) and (9)
show.
Although the SprayAndWaitRouter protocol is also based replication, the fact
that it limits the number of copies that each message may have on the system
make it much more robust in low-buffer scenarios. This contributes to facts
(4), (8) and (9).

The fact that SprayAndWaitRouter makes use of some degree of message
replication while DirectDeliveryRouter does not explains fact (9), since
replication should usually lead to higher delivery probability.
Given that the EpidemicRouter protocol doesn't rely on any historical data
about network topology to make routing decisions, it is heavily dependent on
the speed with which it may flood the network with the packets to be
transmitted. That being the case, improving the transmission speed will usually
improve the delivery probability of such protocol, as fact (5) reveals.

Since the EpidemicOracleRouter protocol - in contrast with the EpidemicRouter
it is based upon - may take information about the contacts that hosts may have
made with each other in the past into account for performing routing decisions,
it is less dependent on flood speed than EpidemicRouter, as fact (6) suggests.
It also tends to lead to more efficient use of message buffers than the latter
protocol and, thus, higher delivery probability, as fact (7) indicates.

The set of tests that we described above were useful for describing each
routing protocols should be used for several different scenarios involving
VANET's. We see that the EpidemicOracleRouter should be the routing
protocol of choice whenever available message buffers are not too small and
higher network utilization is not a great concern.


%% References
%%
%% Following citation commands can be used in the body text:
%% Usage of \cite is as follows:
%%   \cite{key}          ==>>  [#]
%%   \cite[chap. 2]{key} ==>>  [#, chap. 2]
%%   \citet{key}         ==>>  Author [#]

%% References with bibTeX database:

%% \bibliographystyle{model1-num-names}
%% \bibliography{sample.bib}

%% Authors are advised to submit their bibtex database files. They are
%% requested to list a bibtex style file in the manuscript if they do
%% not want to use model1-num-names.bst.

%% References without bibTeX database:

% \begin{thebibliography}{00}

%% \bibitem must have the following form:
%%   \bibitem{key}...
%%

% \bibitem{}

% \end{thebibliography}


\end{document}

%%
%% End of file `elsarticle-template-1-num.tex'.
