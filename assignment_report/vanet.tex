\section{VANET-like Scenario}

While designing the following experiments, we aimed to simulate how DTNs based
on several different routing protocols would behave while serving hosts that
moved like cars. In such scenario, hosts travel along the roads of an accurate
model of Finland's capital, Helsinki, with transmission range set to 50m and
speed varying from 30 to 60 kilometers per hour. Such values were chosen so as
to be representative of the behavior of real cars in a city environment and of
the features that could expect from wireless transmission protocols to be
rolled out in the following years.

\subsection{Experimental Settings}

Tests were described by 3 configuration files generating 30 different parameter
combinations each.  All tests were run once with the same random seed.  The
following tables describe the relevant parameters explored in this scenario.
The DTN performance for each particular configuration was measured according to
a single performance scale that we discuss next.

\paragraph{Performance Scale}

The experiments that we described should be useful for characterizing the
situations in which DTNs could be successfully used as VANETs. That being the
case, we found the need for developing a single performance for uniformly
judging the gathered results, which can be defined as follows. For every
transmission scenario $T_i$ there are measurements for Message Delivery
Probability ($P_i$), Average Latency ($L_i$) and Overhead Ratio ($O_i$):

\begin{itemize}
    \item The Message Delivery Probability is the chance that a message
    created withing simulation time will be delivered to the target host
    before the simulation times out

    \item The Average Latency is the average amount of time, in seconds,
    that it took for the successfully delivered messages to be delivered

    \item The Overhead Ratio is the ratio between the number of packet
    transmissions and the number of messages created by the whole system.
    This value will always be in $[1, \infty[$, where lower figures correspond
    to more efficient use of the transmission channels
\end{itemize}

Based on that, every scenario $T_i$ is assigned a quality grade $Q_i = (p_i,
l_i, o_i) = (\left \lfloor{P_i * 20}\right \rfloor, - log_{2}(L_i), - \left \lfloor{O_i}\right \rfloor)$.
We say that $T_i$ is better than $T_j$ if:

\begin{enumerate}
    \item $p_i > p_j$ or
    \item $p_i = p_j$ and $l_i > l_j$ or
    \item $p_i = p_j$ and $l_i = l_j$ and $o_i > o_j$
\end{enumerate}

We believe this is a meaningful performance scale because Delivery
Probability is the most important performance metric we have under analysis,
followed by Average Latency and Overhead Ratio.

\paragraph{Parameters}

Table \ref{tab:params} lists the parameters used for each configuration
in the experiments.

\begin{table}[htpb]
\centering
    \begin{tabular}{@{}p{0.46\textwidth}p{0.46\textwidth}@{}}
        \toprule
        \multicolumn{2}{c}{\textit{Experimental Configurations}} \\ \midrule
        \textit{Transmission Range} &  50 m \\
        \addlinespace
        \textit{Avg. Packet Creation Time} & 30 s  \\
        \addlinespace
        \textit{Transmission Speed} & 128 KB/s, 1 MB/s \\
        \addlinespace
        \textit{Protocols} & SprayAndWaitRouter, DirectDeliveryRouter, EpidemicRouter, EpidemicOracleRouter \\
        \addlinespace
        \textit{Hosts} &  32, 256, 2048 \\
        \addlinespace
        \textit{Buffer Size} & 1 M, 64 M, 2048 M \\
        \addlinespace
        \textit{Packet TTL} &  5 h \\
        \addlinespace
        \textit{Simulation Time} &  50 min\\ \bottomrule
    \end{tabular}
    \caption{Parameters in tested configurations}
    \label{tab:params}
\end{table}

\subsection{Results}

Tables \ref{tab:res1} and \ref{tab:res2} show the results of all the 72 tests
that were conducted for the scenario above, ordered according to the
performance scale described from best to worst.

\begin{table}[htpb]
\centering
\begin{tabular}{@{}p{0.02\textwidth}p{0.3\textwidth}p{0.06\textwidth}p{0.08\textwidth}p{0.13\textwidth}p{0.04\textwidth}p{0.06\textwidth}p{0.1\textwidth}@{}}
\toprule
\textnumero & Routing Protocol & Hosts & Buffer & T. Speed & P & L & O \\ \midrule
0 & EpidemicOracleRouter & 2048 & 2048M & 128k & 20 & -5 & -3005 \\
1 & EpidemicOracleRouter & 2048 & 2048M & 1M & 20 & -5 & -3005 \\
2 & EpidemicOracleRouter & 2048 & 64M & 128k & 20 & -5 & -37272 \\
3 & EpidemicOracleRouter & 2048 & 64M & 1M & 20 & -5 & -37272 \\
4 & EpidemicRouter & 2048 & 2048M & 1M & 18 & -7 & -2161 \\
5 & EpidemicRouter & 2048 & 64M & 1M & 18 & -7 & -2843 \\
6 & EpidemicRouter & 256 & 2048M & 1M & 16 & -8 & -252 \\
7 & EpidemicRouter & 256 & 64M & 1M & 16 & -8 & -253 \\
8 & EpidemicOracleRouter & 256 & 2048M & 128k & 16 & -8 & -353 \\
9 & EpidemicOracleRouter & 256 & 2048M & 1M & 16 & -8 & -353 \\
10 & EpidemicOracleRouter & 256 & 64M & 128k & 16 & -8 & -363 \\
11 & EpidemicOracleRouter & 256 & 64M & 1M & 16 & -8 & -363 \\
12 & EpidemicRouter & 2048 & 2048M & 128k & 14 & -9 & -2042 \\
13 & EpidemicRouter & 2048 & 64M & 128k & 14 & -9 & -2042 \\
14 & SprayAndWaitRouter & 256 & 64M & 1M & 10 & -9 & -9 \\
15 & SprayAndWaitRouter & 256 & 2048M & 1M & 10 & -9 & -9 \\
16 & EpidemicRouter & 256 & 64M & 128k & 9 & -9 & -174 \\
17 & EpidemicRouter & 256 & 2048M & 128k & 9 & -9 & -174 \\
18 & SprayAndWaitRouter & 256 & 1M & 1M & 7 & -9 & -13 \\
19 & EpidemicOracleRouter & 32 & 64M & 1M & 7 & -10 & -29 \\
20 & EpidemicOracleRouter & 32 & 2048M & 1M & 7 & -10 & -29 \\
21 & EpidemicOracleRouter & 32 & 64M & 128k & 7 & -10 & -29 \\
22 & EpidemicOracleRouter & 32 & 2048M & 128k & 7 & -10 & -29 \\
23 & SprayAndWaitRouter & 2048 & 2048M & 1M & 6 & -9 & -14 \\
24 & SprayAndWaitRouter & 256 & 64M & 128k & 6 & -9 & -14 \\
25 & SprayAndWaitRouter & 256 & 2048M & 128k & 6 & -9 & -14 \\
26 & SprayAndWaitRouter & 2048 & 64M & 1M & 6 & -9 & -14 \\
27 & SprayAndWaitRouter & 2048 & 1M & 1M & 6 & -9 & -15 \\
28 & SprayAndWaitRouter & 32 & 64M & 1M & 6 & -10 & -9 \\
29 & SprayAndWaitRouter & 32 & 2048M & 1M & 6 & -10 & -9 \\
30 & EpidemicRouter & 32 & 64M & 1M & 6 & -10 & -12 \\
31 & EpidemicRouter & 32 & 2048M & 1M & 6 & -10 & -12 \\
32 & SprayAndWaitRouter & 2048 & 64M & 128k & 5 & -10 & -19 \\
33 & SprayAndWaitRouter & 2048 & 2048M & 128k & 5 & -10 & -19 \\
34 & SprayAndWaitRouter & 2048 & 1M & 128k & 4 & -10 & -23 \\
35 & DirectDeliveryRouter & 32 & 2048M & 1M & 3 & -9 & 0 \\ \bottomrule
\end{tabular}
\caption{First part of ordered results}
\label{tab:res1}
\end{table}

\begin{table}[htpb]
\centering
\begin{tabular}{@{}p{0.02\textwidth}p{0.3\textwidth}p{0.06\textwidth}p{0.08\textwidth}p{0.13\textwidth}p{0.04\textwidth}p{0.06\textwidth}p{0.1\textwidth}@{}}
\toprule
\textnumero & Routing Protocol & Hosts & Buffer & T. Speed & P & L & O \\ \midrule
36 & DirectDeliveryRouter & 2048 & 2048M & 1M & 3 & -9 & 0 \\
37 & DirectDeliveryRouter & 32 & 64M & 1M & 3 & -9 & 0 \\
38 & DirectDeliveryRouter & 2048 & 1M & 1M & 3 & -9 & 0 \\
39 & DirectDeliveryRouter & 2048 & 64M & 1M & 3 & -9 & 0 \\
40 & EpidemicRouter & 32 & 64M & 128k & 3 & -10 & -7 \\
41 & SprayAndWaitRouter & 32 & 2048M & 128k & 3 & -10 & -7 \\
42 & SprayAndWaitRouter & 32 & 64M & 128k & 3 & -10 & -7 \\
43 & EpidemicRouter & 32 & 2048M & 128k & 3 & -10 & -7 \\
44 & DirectDeliveryRouter & 2048 & 64M & 128k & 2 & -9 & 0 \\
45 & DirectDeliveryRouter & 256 & 1M & 1M & 2 & -9 & 0 \\
46 & DirectDeliveryRouter & 256 & 2048M & 1M & 2 & -9 & 0 \\
47 & DirectDeliveryRouter & 2048 & 1M & 128k & 2 & -9 & 0 \\
48 & DirectDeliveryRouter & 256 & 64M & 1M & 2 & -9 & 0 \\
49 & DirectDeliveryRouter & 2048 & 2048M & 128k & 2 & -9 & 0 \\
50 & SprayAndWaitRouter & 256 & 1M & 128k & 2 & -9 & -33 \\
51 & EpidemicOracleRouter & 256 & 1M & 128k & 2 & -9 & -376 \\
52 & EpidemicOracleRouter & 256 & 1M & 1M & 2 & -9 & -376 \\
53 & EpidemicRouter & 256 & 1M & 1M & 2 & -9 & -532 \\
54 & DirectDeliveryRouter & 32 & 1M & 1M & 1 & -7 & 0 \\
55 & SprayAndWaitRouter & 32 & 1M & 1M & 1 & -8 & -14 \\
56 & DirectDeliveryRouter & 256 & 2048M & 128k & 1 & -9 & 0 \\
57 & DirectDeliveryRouter & 256 & 1M & 128k & 1 & -9 & 0 \\
58 & DirectDeliveryRouter & 256 & 64M & 128k & 1 & -9 & 0 \\
59 & EpidemicRouter & 2048 & 1M & 1M & 1 & -9 & -60068 \\
60 & DirectDeliveryRouter & 32 & 2048M & 128k & 1 & -10 & 0 \\
61 & DirectDeliveryRouter & 32 & 64M & 128k & 1 & -10 & 0 \\
62 & EpidemicOracleRouter & 32 & 1M & 128k & 0 & -7 & -39 \\
63 & EpidemicOracleRouter & 32 & 1M & 1M & 0 & -7 & -39 \\
64 & EpidemicOracleRouter & 2048 & 1M & 1M & 0 & -7 & -52720 \\
65 & EpidemicOracleRouter & 2048 & 1M & 128k & 0 & -7 & -52720 \\
66 & DirectDeliveryRouter & 32 & 1M & 128k & 0 & -8 & 0 \\
67 & SprayAndWaitRouter & 32 & 1M & 128k & 0 & -8 & -33 \\
68 & EpidemicRouter & 32 & 1M & 1M & 0 & -8 & -36 \\
69 & EpidemicRouter & 32 & 1M & 128k & 0 & -8 & -81 \\
70 & EpidemicRouter & 256 & 1M & 128k & 0 & -9 & -952 \\
71 & EpidemicRouter & 2048 & 1M & 128k & 0 & -9 & -40119 \\ \bottomrule
\end{tabular}
\caption{Second part of ordered results}
\label{tab:res2}
\end{table}

\subsection{Discussion}

Only the EpidemicOracleRouter and the EpidemicRouter protocols are able to
display Delivery Probabilities higher than 70\% for any simulation scenario.
EpidemicOracleRouter and EpidemicRouter are usually much more taxing on the
transmission network than all other routing protocols: if scenarios based on
SprayAndWaitRouter and SprayAndWaitRouter never display overhead ratios above
33 (which occurs for experiment \#50), the same metric is as high as 3005 for the
best performing EpidemicOracleRouter scenario or as high as 37272 for other
top five scenarios using the same protocol, with EpidemicRouter scenarios
showing similar figures.

When all other parameters are fixed, increasing buffer size will usually
improve delivery probability.  For the smallest buffer size, EpidemicRouter and
EpidemicOracleRouter protocols perform the worst, while SprayAndWaitRouter
gives the best results.

For the EpidemicRouter protocol, increasing the transmission speed from 128K to
1M while maintaining all other parameters always improved delivery probability
(see, for example, \#12 vs \#4 and \#16 vs \#7).
For the EpidemicOracleRouter protocol, increasing the transmission speed from
128K to 1M while maintaining all other parameters never improved delivery
probability (see, for example, \#0 vs \#1 and \#10 vs \#11).
With all other parameters kept stable, changing the routing protocol from
EpidemicRouter to EpidemicOracleRouter would usually improve delivery
probability (see, for example, \#30 vs \#19 and \#4 vs \#1).

With transmission speed set to 1 Mbps and buffer size set to 1MB, changing the
routing protocol from either EpidemicRouter or EpidemicOracleRouter to
DirectDeliveryRouter always lead to higher delivery probability or much lower
overhead ratio with similar delivery probability (see, for example, \#63 vs
\#54 and \#59 vs \#38).
With buffer size set to 1MB, changing the protocol from either
EpidemicRouter, EpidemicOracleRouter or DirectDeliveryRouter to
SprayAndWaitRouter would usually lead to much higher delivery probability (see,
for example, \#52 vs \#18 or \#47 vs \#34)

Since EpidemicRouter and EpidemicOracleRouter are protocols that rely heavily
on flooding - not bothering too much with keeping resource utilization down -
these protocols should usually lead to the highest delivery probabilities and
lowest latencies. This also explains why lower
buffer sizes tend to disfavor these protocols.

Although the SprayAndWaitRouter protocol is also based in replication, the fact
that it limits the number of copies that each message may have on the system
make it much more robust in low-buffer scenarios.

The fact that SprayAndWaitRouter makes use of some degree of message
replication while DirectDeliveryRouter does not the higher Delivery Probability
of SprayAndWaitRouter, since replication should usually lead to higher delivery
probability.  Given that the EpidemicRouter protocol doesn't rely on any
historical data about network topology to make routing decisions, it is heavily
dependent on the speed with which it may flood the network with the packets to
be transmitted. That being the case, improving the transmission speed will
usually improve the delivery probability of such protocol.

Since the EpidemicOracleRouter protocol - in contrast with the EpidemicRouter
it is based upon - may take information about the contacts that hosts may have
made with each other in the past into account for performing routing decisions,
it is less dependent on flood speed than EpidemicRouter.
It also tends to lead to more efficient use of message buffers than the latter
protocol and, thus, higher delivery probability.
